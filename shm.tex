\documentclass[a4paper, 12pt]{book}
\usepackage{graphicx}
\usepackage{wrapfig}
\usepackage{index}
\usepackage{xepersian}

\makeatletter
\renewcommand\@endpart{\vfil
              \if@twoside
                \null
                \thispagestyle{empty}%
                \newpage
              \fi
              \if@tempswa
                \twocolumn
              \fi}
\makeatother

\settextfont{B Yekan}
\setlatintextfont[Scale=1.2]{Times New Roman}

\title{ مقدمه ای بر گرافیک کامپیوتر}
\author{احمد منصوری و \lr{\textsf{\textbf{peter shirley}}}}
\date{\lr{December 21, 2009}}

\begin{document}
\maketitle
\let\cleardoublepage\clearpage

\huge\textbf{چکیده}
\normalsize
\begin{flushright}
  همزمان با پیدایش کامپیوتر ها، تلاش ها برای بهره بردن از توان آنها برای ابزار های \lr{Visualize} و دیگر ابزار ها برای 
  استفاده از این قابلیت در زمینه های نظامی، فیلم و انیمیشن، شبیه سازی و بازی سازی شروع شد، این ابزار ها یا به صورت 
  اختصاصی برای استفاده در صنایع خاص طراحی می شوند یا به صورت عمومی تر برای کاربرد های وسیع تری طراحی و توسعه می یابند.
   طراحی و پیاده سازی موتور های گرافیکی به صورت کلی دارای پایه ها و دانشی یکسان از نحوه کار پردازنده ها و ریاضیات است.
   
\end{flushright}


\tableofcontents

\makeatletter
\renewcommand\@endpart{\vfil
              \if@twoside
                \null
                \thispagestyle{empty}%
                \newpage
              \fi
              \if@tempswa
                \twocolumn
              \fi}
\makeatother

\part{گرافیک}

\huge
    مقدمه \\
    
\vspace*{2cm}
\normalsize
     در بخش اول به معرفی و توضیح قسمت های مربوط به \textbf{تصویر} در پروژه می پردازم، این قسمت متشکل از بخش هایی است که
     این قسمت از موتور مسئولیت دریافت مدل های سه بعدی و اطلاعات مربوطه(\lr{\textbf{Textures}, \textbf{Normals}, \textbf{geometry}, ...})
     و نمایش آن ها در صفحه را برعهده دارد، در این قسمت ما با استفاده از ریاضیات مدل ها را در فضای سه یعدی شبیه سازی می کنیم.

\chapter{\lr{Renderer}}


\section{\lr{Texture}}
\begin{figure}[ht]
    \centering
    \includegraphics[width=12cm]{F:/project/shm_build/bin/screenshot.png}
    \caption{\lr{\textit{object without Texture}}}
    \label{fig:my_label}
\end{figure}
\pagebreak


\part{فیزیک}
\begin{center}
    \chapter{\lr{Physics}}
\end{center}
\section{\lr{Particles}}
\section{\lr{Rigid Bodies}}

\end{document} 